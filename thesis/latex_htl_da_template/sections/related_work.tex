\section{Commercial Dart Scoring Systems}

\subsection{High-End Multi-Camera Systems}

\subsubsection{Technical Capabilities}

Professional automatic scoring solutions used in televised tournaments rely on multi-camera triangulation systems. 
These systems typically operate with at least three synchronized high-speed cameras positioned around the board.

Comparable approaches are commonly used in sports tracking research where fast projectiles must be reconstructed in 3D space \cite{Osadchy2006}.
The cameras operate at high frame rates to minimize motion blur and allow precise localization of small objects with high velocity.

To achieve robustness against occlusion, model-based tracking combined with Kalman filtering is applied to estimate the dart trajectory even when the projectile is temporarily invisible \cite{Welch}. 
Sub-pixel localization techniques and geometric board models are then used to classify the hit region.

\subsubsection{Cost and Setup Requirements}

Such systems require calibrated camera arrays and rigid mounting geometry. 
Precise extrinsic calibration is mandatory because triangulation error increases quadratically with baseline misalignment \cite{Hartley2004}. 
Installation therefore requires mechanical fixtures, controlled lighting and dedicated computing hardware.

Consequently, these solutions are expensive and unsuitable for private environments. 
The hardware requirements and calibration complexity motivate the investigation of single-camera alternatives.


\subsection{Laser and Sensor-Based Systems}

\subsubsection{Hardware Dependencies}

Alternative commercial products use laser curtains or vibration sensors attached to the dartboard. 
The dart position is estimated either by beam interruption timing or by analyzing vibration propagation across the board surface.

Comparable sensing principles are used in impact localization research where multiple sensors estimate impact location using time-of-arrival differences \cite{Schwingshackl2010}. 
However, such approaches require specialized hardware integrated into the board.

\subsubsection{Accuracy Characteristics}

While sensor systems work reliably for steel darts, they suffer from material dependency and mechanical damping effects. 
Impact wave propagation varies depending on board wear and mounting conditions, causing systematic localization error \cite{Schwingshackl2010}. 
Furthermore, these methods cannot detect bounce-outs or mid-air collisions and provide no visual verification.

---

\section{Research on Sports Object Tracking}

\subsection{Ball and Projectile Tracking}

\subsubsection{Motion Segmentation Techniques}

Motion segmentation is commonly used to isolate moving objects in video streams. 
Background subtraction methods such as adaptive Gaussian mixture models separate foreground motion from static scenes \cite{Zivkovic2004}. 
These techniques are widely applied in sports tracking to detect fast moving balls or projectiles.

Edge detection and contour extraction further refine object localization by identifying high-contrast boundaries \cite{Canny1986}. 
For circular objects, Hough transform based detection is frequently used \cite{Duda1972}. 
These techniques form a computationally efficient alternative to deep learning approaches when the environment is controlled.

\subsubsection{Trajectory Estimation Methods}

Once detected, object motion is typically modeled using recursive state estimation. 
Kalman filtering predicts the projectile path and compensates for measurement noise \cite{Welch}. 
In sports applications this enables recovery of temporarily occluded objects and estimation of impact position \cite{Osadchy2006}.

The combination of motion detection and trajectory prediction is particularly effective for high-contrast fast moving objects where training data for neural networks is limited.

---

\section{Existing Open Source Projects}

\subsection{Dartboard Detection Tools}

\subsubsection{Marker-Based Alignment}

Several research prototypes rely on fiducial markers for camera calibration. 
ArUco markers allow reliable pose estimation even under partial occlusion. 
By detecting multiple markers attached to the board, a homography between image and board coordinates can be computed.

This allows transforming detected impact points into standardized dartboard coordinates independent of camera perspective.

\subsubsection{Geometric Modeling Approaches}

After calibration, most systems apply a geometric dartboard model consisting of concentric rings and radial sectors. 
The hit location is classified by computing the nearest ring boundaries and angular sector. 
Such geometric classification avoids machine learning and instead relies on deterministic mapping from pixel coordinates to board regions.

This approach is computationally efficient and well suited for real-time applications on consumer hardware.

---

\section{Position of This Work in Current Literature}

\subsection{Comparison with Existing Approaches}

\subsubsection{Hardware Differences}

Existing systems either depend on expensive multi-camera setups or dedicated sensor hardware. 
The presented work instead uses a single consumer camera and passive board markers. 
Compared to stereo systems \cite{Hartley2004}, no multi-view calibration is required, reducing installation effort.

Compared to vibration-based systems \cite{Schwingshackl2010}, the method remains independent of board material and mounting conditions.

\subsubsection{Novelty of a Motion-Threshold Pipeline}

The proposed pipeline combines motion segmentation, temporal thresholding and geometric board modeling. 
Unlike neural network approaches, the system does not require training data. 
Instead it relies on classical computer vision techniques including background subtraction \cite{Zivkovic2004}.

This positions the work between hardware-heavy commercial solutions and data-hungry AI methods. 
The contribution lies in demonstrating that reliable dart detection can be achieved using a deterministic single-camera pipeline suitable for private use environments.
