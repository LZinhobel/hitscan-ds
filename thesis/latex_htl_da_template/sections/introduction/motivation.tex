\section{Motivation}

Darts is a sport that blends precision, concentration, and social interaction, enjoyed by millions worldwide in both professional arenas and amateur settings like pubs, clubs, and private homes. The digital age has transformed many aspects of recreational sports, offering enhanced experiences through data tracking and automated processes. However, the sport of darts, particularly at the amateur level, remains largely analog in its core gameplay loop.

\subsection{Challenges in Amateur Dart Scoring}

\subsubsection{Lack of Affordable Automatic Systems}

Apart from soft-tip dart systems with integrated electronic boards, no affordable automatic scoring solution exists for traditional steel-tip darts. While electronic soft-tip boards provide built-in hit detection, they fundamentally alter the playing experience and are therefore not accepted as a substitute by many players who prefer the physical characteristics and authenticity of classic bristle boards.

High-end commercial solutions such as \textit{\textbf{Scolia}} and \textit{\textbf{Autodarts}} provide automatic scoring capabilities for steel darts. However, these systems require multiple cameras, permanent mounting installations, and specialized hardware components. As a result, they operate in a significantly higher price range and are primarily targeted at dedicated training environments or semi-professional users rather than casual home players.

Consequently, a clear market gap emerges: players must currently choose between inexpensive traditional dartboards with no digital functionality and costly automated systems requiring complex installation. A low-cost retrofit solution capable of augmenting existing analog dartboards with automatic scoring functionality is currently missing, limiting accessibility of modern scoring technology to a large portion of the player base.

\subsubsection{Limitations of Manual Score Entry}

The dominant alternative to automatic detection in steel-tip darts is manual scorekeeping. This process exists in several variants: players may calculate scores using a calculator, record them in a mobile application, or write them manually on a whiteboard or chalkboard. Regardless of the method, the player must repeatedly interrupt the game, leave the throwing position, and update the score after each visit.

This interruption negatively affects concentration and game flow. The necessity to mentally switch from motor coordination to arithmetic calculation introduces cognitive load and often slows down the overall playing experience. Furthermore, manual scorekeeping is inherently error-prone, particularly during complex checkout calculations, which can lead to disputes and frustration.

In addition, manual systems provide no automatic collection of performance data. Players cannot easily track averages, checkout percentages, hit distributions, or long-term consistency. As research in sports analytics shows, the availability of performance statistics significantly contributes to motivation, learning, and sustained engagement. The absence of automated data collection therefore prevents players using traditional boards from benefiting from analytical feedback that is standard in many modern sports environments.